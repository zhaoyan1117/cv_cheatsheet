\documentclass[10pt,landscape]{article}
\usepackage{multicol}
\usepackage{calc}
\usepackage[landscape]{geometry}
\usepackage{amsmath,amsthm,amsfonts,amssymb,relsize}
\usepackage{color,graphicx,overpic}

% Turn off header and footer
\pagestyle{empty}
\geometry{top=.25in,left=.25in,right=.25in,bottom=.25in}

% Redefine section commands to use less space
\makeatletter
\renewcommand{\section}{\@startsection{section}{1}{\z@}{3ex}{2ex}
                       {\normalfont\normalsize\bfseries\textit}}
\renewcommand{\subsection}{\@startsection{subsection}{2}{\z@}{2ex}{0.5ex}
                          {\normalfont\small\bfseries}}
\makeatother

% Don't print section numbers
\setcounter{secnumdepth}{0}

\setlength{\parindent}{0pt}
\setlength{\parskip}{0pt}

\input{math_commands.tex}

\newcommand{\ruler}{\\\rule{\columnwidth}{0.25pt}\\}

% -----------------------------------------------------------------------
\begin{document}
\raggedright
\footnotesize
\begin{multicols*}{3}
% multicol parameters
% These lengths are set only within the two main columns
% \setlength{\columnseprule}{0.1pt}
\setlength{\premulticols}{1pt}
\setlength{\postmulticols}{1pt}
\setlength{\multicolsep}{1pt}
\setlength{\columnsep}{0pt}

\section{Geometry of Image Formation}
Pin hole camera:
$x = \frac{-fX}{Z}, y = \frac{-fY}{Z}$
\ruler
Projection of lines $\bX = \bA + \lambda\bD$:
$\lim_{\lambda\to\infty}\bp=f\frac{\bA+\lambda\bD}{A_Z+\lambda D_Z}=f\frac{\bD}{D_Z}$
\ruler
Projection of planes $\bX\cdot\bN = d$:
$\lim_{Z\to\infty}\bp\cdot\bN=0 \Rightarrow xN_x+yN_y+fN_z=0$
\ruler
Terrestrial perspective:\\
Ground plane has normal $\bN = (0,1,0)$ and projects to $y=0$.\\
Camera height $h_c$; an object of height $\delta Y$ with bottom at $(X, -h_c, Z)$ and
top at $(X, \delta Y - h_c, Z)$. The bottom projects to $(fX/Z, -fh_c/Z)$ and
top projects to $(fX/Z, f(\delta Y - h_c)/Z)$.

\section{Geometric Transformations}
\textbf{Pose}: the position and orientation of the object with respect to the camera.
6 degrees of freedom.\\
\textbf{Shape}: the coordinates of the points of an object relative to a coordinate frame
on the object; invariant when the object undergoes rotations and translations.
\ruler
\textbf{Euclidean Transformations}: rotations and translation are isometries or
regid body motions that preserves distances between any pair of points.
\ruler
\textbf{Orthogonal Transformations}:\\
preserves inner products: $\ba\cdot\bb = \psi(\ba)\cdot\psi(\bb)$\\
preserves norms: $\|\ba\| = \|\psi(\ba)\|$\\
are isometries: $(\psi(\ba)-\psi(\bb))\cdot(\psi(\ba)-\psi(\bb)) = (\ba-\bb)\cdot(\ba-\bb)$
\ruler
\textbf{Orthogonal matrices}:\\
$\tp{\bA} = \inv{\bA}$, $\det(\bA)^2 = 1 \Rightarrow \det(\bA) = +1 \mbox{ or } -1$.\\
$\underbrace{
\begin{bmatrix}
\cos\theta & -\sin\theta\\
\sin\theta & \cos\theta
\end{bmatrix}
}_{\text{rotation, $\det = +1$}}
\underbrace{
\begin{bmatrix}
\cos\theta & \sin\theta\\
\sin\theta & -\cos\theta
\end{bmatrix}
}_{\text{rotation, $\det = -1$}}$
\ruler
\textbf{Group structure of isometries}:\\
$\psi(\ba)=\bA\ba+\bt$, $\bA$ is orthogonal matrix.\\
$\psi_1(\ba)=\bA_1\ba+\bt_1, \psi_2(\ba)=\bA_2\ba+\bt_2$,
$\psi_1\circ\psi_2(\ba)=(\bA_1\bA_2)\ba+(\bA_1\bt_2+\bt_1)$,
\ruler
\textbf{Skew-symmetric matrix}:
$\bS = -\tp{\bS}$\\
\textbf{Cross product Matrix}:\\
$
\begin{bmatrix}
a_1\\
a_2\\
a_3
\end{bmatrix}
\wedge
\begin{bmatrix}
b_1\\
b_2\\
b_3
\end{bmatrix}
=
\begin{bmatrix}
a_2b_3-a_3b_2\\
a_3b_1-a_1b_3\\
a_1b_2-a_2b_1
\end{bmatrix}
\Rightarrow
\hat{\ba}
\defeq
\begin{bmatrix}
0 & -a_3 & a_2\\
a_3 & 0 & -a_1\\
-a_2 & a_1 & 0
\end{bmatrix}
$
\ruler
\textbf{Roderigues Formula}:\\
$\bR = e^{\phi\hat{s}} = \bI + \sin\phi\hat{\bs}+(1-\cos\phi)\hat{\bs}^2$,\\
$\bs$ is a unit vector along $\omega$ and $\phi = \|\omega\|t$ is the total amount of rotation.
\ruler
\textbf{Affine transformations} is a nonsingular linear transformation followed by a translation.
$\psi(\ba)=\bA\ba+\bt$ ($\det \bA \neq 0$) and preserves parallelism and midpoints
and does \textbf{not} preserve lengths, angles, areas.\\
Ex. rotation, anisotropic scaling, sheer.\\
\columnbreak
\textbf{Degrees of freedom}:\\
2D: isometry: $1 + 2 = 3$, affine: $4 + 2 = 6$\\
3D: isometry: $3 + 3 = 6$, affine: $9 + 3 = 12$

\section{Optical Flow}
Motion at 3D world projects to motion in the image. At every point $(x,y)$ in the image we get a
2D vector, corresponding to the motion of the feature located at that point.\\
\textbf{egomotion} is the optical flow field of a moving observer.
\ruler
$\bt$ is translational velocity of camera, $\bomega$ is angular velocity of camera.
$
\begin{bmatrix}
u\\v
\end{bmatrix}
=
\frac{1}{Z}
\begin{bmatrix}
-1 & 0 & x\\
0  & -1 & y
\end{bmatrix}
\begin{bmatrix}
t_x\\t_y\\t_z
\end{bmatrix}
+
\begin{bmatrix}
xy & -(1+x^2) & y\\
1+y^2 & -xy & -x
\end{bmatrix}
\begin{bmatrix}
\omega_x\\\omega_y\\\omega_z
\end{bmatrix}
$

\section{Radiometry of Image Formation}
\textbf{Irradiance}:\\
$I(x,y)$ measures how much light is captured at pixel $(x,y)$.
Radiant power per unit area with units W/$\text{m}^2$, denoted by E.
\ruler
$L = \frac{\text{Power}}{(dAcos\theta)(d\omega)}$, units are $Wm^{-2}sr^{-1}$.
\ruler
\textbf{Lambertian model}: $L = \rho\lambda n \cdot s$

\end{multicols*}
\end{document}
